\chapter{Especificação de requisitos}
\label{CAP4}

Neste capítulo serão descritas as necessidades básicas levantadas para que o projeto funcione como esperado.

\section{Requisitos funcionais}


A tabela de requisitos inicial do projeto pode ser vista na tabela \ref{tab:req_ini}.

    \begin{table}[H]
        \caption{Tabela inicial de requisitos funcionais.}
        \label{tab:req_ini}
        \begin{tabular}{|c|c|}
        \hline
        { \textbf{Propósito}}       & { Estabilizar uma câmera em 3 eixos}                                                                       \\ \hline
        { \textbf{Entradas}}        & { \begin{tabular}[c]{@{}c@{}}Botão de energia, botão de trava,\\ acelerômetro e giroscópio\end{tabular}} \\ \hline
        { \textbf{Saídas}}          & { 3 motores de passo}                                                                                      \\ \hline
        { \textbf{Funções}}         & { Estabilizar uma câmera em tempo real}                                                                    \\ \hline
        { \textbf{Performance}}     & { Ajusta a posição a cada 10ms}                                                                            \\ \hline
        { \textbf{Custo}}           & { R\$250,00}                                                                                               \\ \hline
        { \textbf{Potência}}        & { 1W}                                                                                                      \\ \hline
        { \textbf{Tamanho e massa}} & { Até 200g, aproximadamente 30cm x 5cm x 5cm}                                                              \\ \hline
        \end{tabular}
    \end{table}
        
Ao longo do desenvolvimento diversos requisitos foram alterados para melhor enquadrarem os objetivos de longo prazo do projeto, conforme pode ser visto na tabela \ref{tab:req_fin}.

    \begin{table}[H]
        \caption{Tabela final de requisitos funcionais.}
        \label{tab:req_fin}
        \begin{tabular}{|c|c|}
        \hline
        { \textbf{Propósito}}       & { Estabilizar uma câmera em 3 eixos}                                                                       \\ \hline
        { \textbf{Entradas}}        & { \begin{tabular}[c]{@{}c@{}}Botão de energia, botão de trava,\\ acelerômetro e giroscópio\end{tabular}} \\ \hline
        { \textbf{Saídas}}          & { 3 motores de passo, 4 leds}                                                                                      \\ \hline
        { \textbf{Funções}}         & { Estabilizar uma câmera na taxe de quadros da camera}                                                                    \\ \hline
        { \textbf{Performance}}     & { Ajusta a posição a cada 16,6ms}                                                                            \\ \hline
        { \textbf{Custo}}           & { R\$350,00}                                                                                               \\ \hline
        { \textbf{Corrente}}        & { ~16W}                                                                                                     \\ \hline
        { \textbf{Tamanho e massa}} & { Até 500g, aproximadamente 30cm x 5cm x 5cm}                                                              \\ \hline
        \end{tabular}
    \end{table}


\section{Requisitos não-funcionais}

A seguir são descritos alguns dos requisitos que não são necessários para que o projeto funcione, mas são absolutamente necessários para a aceitação dos \textit{stakeholders}, ou seja, todas as partes envolvidas, incluindo os usuários do produto, os desenvolvedores e as empresas vendedoras.

\begin{itemize}
    \item \textbf{Preço baixo}
        
    Os produtos comerciais de Gimbal, por serem feitos para câmeras profissionais, precisam de mais robustez, por isso, custam mais caro: a linha Gimbal DJI custa acima de R\$2500,00 e a linha S Leeremi acima de R\$1000,00.
        
    O projeto precisa custar consideravelmente menos para que, mesmo após a incidência de impostos e taxas de transporte, o estabilizador tenha preço mais atrativo em relação aos concorrentes.
        
    \item \textbf{Versatilidade}
    
    Este estabilizador servirá para câmeras semi-profissionais, profissionais e também para celulares.
    
    Diferente de linhas como a Zhiyun Smooth, que só servem para celulares, o trabalho procura atender vários tipos de públicos.
    
    
    \item \textbf{Equivalência de captura de imagem}
    
    As imagens geradas usando o estabilizador descrito neste documento devem ter a mesma qualidade das produzidas usando os produtos já disponíveis no mercado.
    
    Isto é extremamente importante para que o projeto seja visto pela sociedade como um produto substituto, e não um de menor qualidade.
    
    
    \item \textbf{Robustez}
    
    Assim como o item anterior, a confiança do consumidor sobre o produto deverá ocorrer graças à qualidade oferecida, mesmo com preço mais baixo.
    
    Dessa forma, é preciso que os materiais usados no estabilizador não quebrem facilmente. A estrutura deve resistir aos choques mecânicos mais comuns, uma vez que, manipulada por seres humanos, sempre pode cair no chão.
\end{itemize}





