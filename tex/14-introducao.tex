\chapter{Introdução}\label{CAP:introducao}

\section{Motivação e Contexto}
A cultura \textit{maker} assemelha-se, e possivelmente se originou da tendência de artesanato DIY(\textit{Do-It-Yourself}), a qual permite que uma pessoa comum possa fabricar seus próprios objetos. A lógica é contrária à industrial, uma vez que, em vez de produtos uniformes, o processo resulta em objetos personalizados.\cite{cultura_maker}.

Os \textit{makers} como existem hoje surgiram recentemente, assim que houve a ampliação do acesso a ferramentas como impressoras 3D e cortadoras a laser, assim como canais de comunicação para compartilhar conhecimento sobre operá-las. Do mesmo modo, a popularização de plataformas de prototipagem como o Arduino e ESP32 os tornou parte integral deste movimento.

Vários exemplos da cultura \textit{maker} podem ser encontrados no \textit{YouTube}, e um exemplo bastante famoso é o Manual do Mundo\cite{manual_do_mundo}, no quadro Manual Maker\cite{manual_maker}. Ele é, inclusive, uma das maiores inspirações para este trabalho.

Levando essa filosofia em conta, este trabalho procura definir um produto que concilia a visão industrial e a \textit{maker}, uma vez que poderá ser produzido em escala, mas as subpartes do objeto final podem ser facilmente achadas por pessoas comuns.

O produto que se quer descrever já existe comercialmente, é um estabilizador Gimbal. O problema do que é vendido é o fato de não ser acessível para o grande público, isto é, o público-alvo desse produto são fotógrafos profissionais. Este trabalho procura, portanto, expandir o mercado já existente.
 
\section{Objetivo}

O objetivo principal desse trabalho é desenvolver uma prova de conceito sobre um sistema de estabilização de câmeras. Provendo tanto o \textit{software} quanto o \textit{hardware} necessários para que outras pessoas possam construí-lo.

Dessa maneira, o sistema deve ser uma base para que outros sistemas de desenvolvimento \textit{maker} possam ser construídos. Um Gimbal, por exemplo, poderia dar origem a vários outros projetos, como orientação de sensores e ajuste de equipamentos.


\section{Justificativa}

O projeto será uma base para futuras expansões, em que outras pessoas podem usar seus conceitos para não só construí-lo, como criar novas soluções através dele. 

O RocketPy\cite{RocketPy} pode ser citado como um exemplo que este projeto busca emular; criado por alunos da Escola Politécnica, como uma ferramenta interna do Projeto Jupiter, essa biblioteca Python se tornou o estado da arte para sistemas de previsão de trajetória dentro do ambiente de foguete-modelismo e pode ser um recurso vital para diversos novos sistemas.

Os produtos mais atuais não incluem apenas sistemas comerciais caros, a iniciativa de código aberto tem expandido bastante. Em 2017, por exemplo, a equipe OHM da Universidade Técnica da Dinamarca desenvolveu um sistema \textit{open-source}, o OpenSAM, mas esse projeto usa um microcontrolador comercial SimpleBGC de 8 bits, feito para estabilização\cite{open_sam}. 

\section{Organização do trabalho}
Os próximos capítulos explicitarão como o trabalho foi planejado a partir de seus requisitos e das tecnologias envolvidas.

O projeto consistirá em montar um estabilizador ativo como projeto de sistema embarcado.

Para isso, o Capítulo 2 faz uma breve apresentação dos conceitos usados neste projeto. O Capítulo 3, por sua vez, traz as etapas de desenvolvimento do trabalho e o Capítulo 4 mostra quais requisitos foram definidos para o sistema. Além disso, o desenvolvimento inicial do trabalho é descrito no Capítulo 5. Por último, o Capítulo 6 traz as considerações finais com uma breve discussão das conclusões do trabalho, assim como contribuições e sugestões para a continuidade dele.
