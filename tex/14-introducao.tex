\chapter{Introdução}\label{CAP:introducao}

Este documento descreve o desenvolvimento de um sistema estabilizador de câmeras que segue o mesmo princípio dos atuais estabilizadores Gimbal.

A ideia básica do projeto é construir a estrutura do estabilizador e, posteriormente, testá-la em câmeras diversas: de celular, semi-profissionais e profissionais.

\section{Motivação}
A cultura \textit{maker} consiste em ...\textsuperscript{\cite{cultura_maker}}

 
\section{Objetivo}




\section{Justificativa}


\section{Organização do trabalho}
Os próximos capítulos explicitarão como o trabalho foi planejado a partir de seus requisitos e das tecnologias envolvidas.

O projeto consistirá em montar um estabilizador ativo como projeto de sistema embarcado.

Para isso, o Capítulo 2 faz uma breve apresentação dos conceitos usados neste projeto. O Capítulo 3, por sua vez, traz as etapas de desenvolvimento do trabalho e o Capítulo 4 mostra quais requisitos foram definidos para o sistema. Além disso, o desenvolvimento inicial do trabalho é descrito no Capítulo 5. Por último, o Capítulo 6 traz as considerações finais com uma breve discussão das conclusões do trabalho, assim como contribuições e sugestões para a continuidade dele.
