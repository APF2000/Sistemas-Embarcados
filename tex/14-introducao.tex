\chapter{Introdução}\label{CAP:introducao}

Este documento descreve o desenvolvimento de um sistema estabilizador de câmeras que segue o mesmo princípio dos atuais estabilizadores Gimbal.

A ideia básica do projeto é construir a estrutura do estabilizador e, posteriormente, testá-la em câmeras diversas: de celular, semi-profissionais e profissionais.

\section{Motivação}
A cultura \textit{maker} consiste em permitir que uma pessoa comum possa fabricar seus próprios objetos. A lógica da cultura \textit{maker} é justamente contrária à industrial, uma vez que, em vez fabricar produtos uniformes, em escala, em uma linha de produção, a fabricação torna-se personalizada e feita pelo próprio usuário\cite{cultura_maker}.

Levando essa filosofia em conta, este trabalho procura definir um produto que concilia a visão industrial e a cultura \textit{maker}, uma vez que poderá ser produzido em escala, mas as subpartes do objeto final podem ser facilmente achadas por pessoas comuns.

O produto que se quer descrever já existe comercialmente, é um estabilizador Gimbal.
 
\section{Objetivo}

O objetivo principal desse trabalho é desenvolver uma prova de conceito um sistema de estabilização de uma câmera por gimbal. Provendo tanto o software quanto o hardware necessário que outras pessoas possam construí-lo.

Dessa maneira, o sistema deve ser uma base para que outros sistemas de desenvolvimento maker possam ser construidos, um gimbal por exemplo tem vários como orientação de câmeras e sensores, ajustes de equipamentos.


\section{Justificativa}

O projeto será uma base para futuras expansões, em que outras pessoas podem usas seus conceitos para não só construí-lo, como criar novas soluções através dele. 

O RocketyPy, pode ser citado como um exemplo que esse projeto busca emular, criado por alunos da Escola Politécnica, como uma ferramenta interna do Projeto Jupiter, a biblioteca Python se tornou o estado da arte para sistemas de previsão de trajetória dentro do ambiente de foguete-modelismo e é a base ou um recurso vital para diversos novos sistemas.

O estado da arte não só já conta com sistemas comerciais caros, em 2017 a equipe OHM da universidade tecnicas da dinamarca desenvolveu um sistema open-source o OpenSAM, mas esse projeto usa um microcontrolador comercial SimpleBGC de 8 bits, feito para estabilização. 

\section{Organização do trabalho}
Os próximos capítulos explicitarão como o trabalho foi planejado a partir de seus requisitos e das tecnologias envolvidas.

O projeto consistirá em montar um estabilizador ativo como projeto de sistema embarcado.

Para isso, o Capítulo 2 faz uma breve apresentação dos conceitos usados neste projeto. O Capítulo 3, por sua vez, traz as etapas de desenvolvimento do trabalho e o Capítulo 4 mostra quais requisitos foram definidos para o sistema. Além disso, o desenvolvimento inicial do trabalho é descrito no Capítulo 5. Por último, o Capítulo 6 traz as considerações finais com uma breve discussão das conclusões do trabalho, assim como contribuições e sugestões para a continuidade dele.
